\documentclass[11pt,a4paper]{article}
 
%-----------MATEMATYKA----------------------------------------
\usepackage{array}
\usepackage{amsmath}
\usepackage{amssymb}
%-------------------------------------------------------------

%\usepackage{multiline}
\usepackage{changepage}
\usepackage{geometry}
\newgeometry{tmargin=2.5cm, bmargin=3.5cm, lmargin=3.5cm, rmargin=3.5cm}

\usepackage{indentfirst}
\usepackage{graphicx} 	
\usepackage{float}
\usepackage[utf8]{inputenc} 
\usepackage[english,polish]{babel}
\usepackage{polski}

%------------ HYPERREF----------------------------------------
\usepackage{hyperref}
% tu można dodać \usepackage{caption} dla figure gez caption
\usepackage[all]{hypcap} % odnośniki do góry figure

\usepackage{xcolor}  
\hypersetup{         %stylowe hyperrefy zamiast czerwonych ramek
    colorlinks,
    linkcolor={red!50!black},
    citecolor={blue!50!black},
    urlcolor={blue!80!black}
}
%-------------------------------------------------------------


\title{Teoria optymalizacji - sprawozdanie z projektu}
\author{Piotr Dulewicz 209253 \\ Konrad Pleban numerindeksu \\ Termin zajęć: poniedziałek parzysty 11:15 \(-\) 13:00}
\date{}

\begin{document}
\maketitle
\section{Sformułowanie zadania optymalizacji}
sample text

\section{Omówienie algorytmu optymalizacji}
sample text

\section{Informacje o programie}
Program został napisany w języku C++, z użyciem biblioteki graficznej Qt, w środowisku Qt Creator. Do wizualizacji zbioru Pareto w przestrzeni funkcyjnej użyta została klasa QCustomPlot. Jej autorem jest Emanuel Eichhammer. Narzędzie to udostępnione jest na licencji GNU GPL pod adresem \url{http://www.qcustomplot.com/index.php/}. Do interpretacji wyrażeń matematycznych skorzystano z biblioteki \textit{C++ Mathematical Expression Toolkit}. Jej autorem jest Arash Partow. Biblioteka jest dostępna na licencji MIT pod adresem 
\url{http://partow.net/programming/exprtk/index.html}. 

\section{Wprowadzanie danych początkowych}
sample text

\section{Przykłady testowe}
sample text

\begin{thebibliography}{9}
\bibitem{pa} K. Deb and A. Pratap and S. Agarwal and T. Meyarivan:
\emph{A Fast and Elitist Multiobjective Genetic Algorithm: NSGA–II},
IEEE Transactions on Evolutionary Computation, 2002
\end{thebibliography}

\end{document}